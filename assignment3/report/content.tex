\section{需求分析}

本程序要完成的功能为,判断字符串是否为回文字符串。
输入要求是一串字符串,并且要以\#结束。
例如,
avava\#。
如果有必要,可以读入多个字符串串。
ava\#abaaba\#。
我们对输入做出以下限制:
\begin{enumerate}
   \item 输入的串的总长度不能超过$10^6$
   \item 输入的所有字符只能在[0~9][a~z][A~Z]中\#
\end{enumerate}


在正常情况下,本程序输出T或F判断正误。在有多个字符串输入的情况下,
判断结果将按顺序输出。
以上面两个输入样例为例。



\section{概要设计}
   问题解决的思路概述;说明程序中用到的所有数据结构类型的定义,主程序的流程以及各程序模块之间的层次(调用)关系。
   
   
   主程序,即在main函数中调用输入和创建初始化数据结构的函数,进行数据处理,最后输出
   结果。


   我们需要定义以下数据结构
   \begin{enumerate}
      \item 队列,只需要支持队列的基本操作
      \item 栈,只需要支持栈的基本操作
      \item 处理数据的solution
   \end{enumerate}


   对于读入非\#字符,main会调用Solution中的insert函数插入数据。
   每当读入一个\#的时候,main中就会调用Solution中check函数判断之前读入的串是否为回文串。

\section{详细设计}

   先分别定义队列和栈数据结构


   队列中含有的元素为:
   \begin{enumerate}
      \item 头指针,head
      \item 尾指针,tail
      \item 插入insert
      \item 弹出数据pop\_tail
      \item 获取数据get\_head
   \end{enumerate}


   同样有栈的定义如下
   \begin{enumerate}
      \item 栈顶指针top
      \item 取得栈顶的数据get\_top
      \item 弹出栈顶的数据pop\_top
   \end{enumerate}

\begin{algorithm}[htb] 
   \caption{ Solution结构定义 } 
   \label{alg:Framwork} 
   \begin{algorithmic}[1]
      \Require 输入串
      \Ensure 判断串是否回文
      
      \State 插入数据函数
      \Function {insert}{a}
         \If {a 不合法}
            \State \Return FAILURE
         \EndIf
         \State 将a添加到栈中
         \State 将a添加到队列中
         \State \Return SUCCESS
      \EndFunction \\

      \State 检查是否为回文串函数
      \Function {check}{void}
         \While{栈和队列非空}
            \State 从栈中取出一个元素stack\_data
            \State 从队列中取出一个元素queue\_data
            \If{$stack\_dat \neq queue\_data$}
               \State \Return FAILURE
            \EndIf
            \State 从栈和队列中弹出数据
         \EndWhile
         \State \Return SUCCESS
      \EndFunction
   \end{algorithmic} 
   \end{algorithm}

\section{调试分析报告}

\section{用户使用说明}

\section{测试结果}


