\section{需求分析}

本程序要完成的功能为,将一段二进制序列转化为八进制的序列。
输入要求01序列,并且要以\#结束。
例如,
101010011010\#。
如果有必要,可以读入多个二进制串。
100110\#1010101\#。
我们对输入做出以下限制:
\begin{enumerate}
   \item 输入的串的长度不能超过$10^6$
   \item 输入的所有字符只能有0,1,\#
\end{enumerate}


在正常情况下,本程序输出一段8进制串。在有多个二进制串输入的情况下,
八进制串将按顺序输出。
以上面两个输入样例为例。



\section{概要设计}
   main创建并初始化Solution结构,并且循环调用Solution中的读入和输出。


   Solution中保存了两个栈,一个用于放置二进制串,一个用于放置八进制串。
   栈中实现了push,pop,get\_top等栈的基本功能。
   Solution结构可以调用readBinary进行读入,调用writeOctal写入。
   再在Solution中实现一个可以将二进制数转化为八进制的中间函数。

\section{详细设计}

\begin{algorithm}[htb] 
   \caption{ Solution结构定义 } 
   \label{alg:Framwork} 
   \begin{algorithmic}[1]
      \Require 输入二进制串
      \Ensure 输出八进制串
      
      \State 读入二进制串
      \Function {readBinary}{void}
         \State 读入字符a
         \While {a不是\#也不是EOF}
            \If {}
            \EndIf
         \EndWhile
      \EndFunction \\

      \State 输出八进制串
      \Function {writeOctal}{void}
         
      \EndFunction

      
      \Function {check}{void}
         
      \EndFunction
   \end{algorithmic} 
\end{algorithm}

\section{调试分析报告}

\section{用户使用说明}

\section{测试结果}


