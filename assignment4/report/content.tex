\section{需求分析}
   本程序可以完成两个矩阵相乘或相减的计算任务。
   输入一个矩阵的格式为
   
   
   $m$ $n$ $p$


   $i_1$ $j_1$ $val_1$


   $i_2$ $j_2$ $val_2$


   $i_3$ $j_3$ $val_3$


   ........


   $i_p$ $j_p$ $val_p$


   $m$表示矩阵的行数,$n$表示矩阵的列数,$p$表示稀疏矩阵中结点的个数。
   

   本程序要求用户输入两个矩阵A、B,然后再选择计算方式。然后程序会在后端进行计算,
   将计算结果保存在矩阵C中。
   最后程序将C输出到终端,输出格式同样使用三元组进行输出,顺序为先列后行。


   同时,我们对于用户的输入作出如下限制:
   \begin{enumerate}
      \item $n > 0$ 且 $m > 0$ 且 $p >= 0$
      \item 对于一个矩阵,数值的下标应该保持在矩阵范围内(从1开始),不得超出矩阵。
      \item 必须要输入两个矩阵后再进行运算。
   \end{enumerate}
   对于不符合输入限制要求的操作,程序将在终端上输出(Input Error)


   同时,用户应该对输入的矩阵是否能进行运算进行检查。
   如果输入的矩阵不能进行运算,程序将返回(Math Error).
   可能的原因分别有:
   \begin{enumerate}
      \item Error Code 0: 相加的两个矩阵的行数与列数不相等。
      \item Error Code 1: 两个矩阵不能相乘
   \end{enumerate}
\section{概要设计}
   程序由三个模块组成,分别为$Matrix$,$Solution$和$main$

   
   $Matrix$ 是一种数据类,定义了稀疏矩阵类以及它的运算方法,包括在矩阵中插入元素。


   $Solution$ 是可以与用户交互的类,定义了如何输入和输出,以及调用$Matrix$中的方法完成计算。


   $main$ 为程序的主函数,主要用于初始化$Solution$和调用$Solution$中与用户交互的方法。
   主程序的流程如下:
   \begin{enumerate}
      \item 初始化$Solution$
      \item 调用输入方法
      \item 调用计算方法
      \item 调用输出方法
      \item 释放内存
   \end{enumerate}

\section{详细设计}
\begin{algorithm}[htb] 
   \caption{ Solution结构定义 } 
   \label{alg:Framwork} 
   \begin{algorithmic}[1]
      \State 读入需要被操作的矩阵
      \Function {readMatrix}{void}
         \State 初始化矩阵A
         \State 读入m,n,size
         \For {$1 \to size$}
            \State 读入三元组,并插入到矩阵A中
         \EndFor

         \State 初始化矩阵B
         \State 读入$m$,$n$,$size$
         \For {$1 \to size$}
            \State 读入三元组,并插入到矩阵B中
         \EndFor
      \EndFunction
      \State 读入运算类型并得出运算结果
      \Function{operate}{void}
         \State 从用户处读入运算类型,保存在op中
         \If {$op = 0$}
            \State   result = X + Y
         \Else
            \State   result = X * Y
         \EndIf
      \EndFunction
      \State 打印运算结果
      \Function{printMatrix}{void}
         \State 输出Result矩阵
      \EndFunction

   \end{algorithmic} 
\end{algorithm}



\begin{algorithm}[htb] 
   \caption{ Solution结构定义 } 
   \label{alg:Framwork} 
   \begin{algorithmic}[1]
      \State 将一个三元组插入到矩阵中
      \Function {insert}{Tuple node}
         \State 检查Matrix是否超限
         \For{Matrix所有的三元组}
            \If{第i个三元组与node在同一个位置}
               \State 将node的数值加到当前三元组数值上
            \EndIf
         \EndFor
         \If{Matrix中没有该三元组}
            \State 在三元组尾部插入node
         \EndIf
      \EndFunction
      \State 将当前矩阵与矩阵X相加,返回结果
      \Function{operator +}{Matrix X}
         \State 检查是否符合相加的条件
         
      \EndFunction
      \State 将当前矩阵右乘矩阵X,返回结果
      \Function{printMatrix}{Matrix X}
         \State 输出Result矩阵
      \EndFunction

   \end{algorithmic} 
\end{algorithm}

\section{调试分析报告}

\section{用户使用说明}

\section{测试结果}



