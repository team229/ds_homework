\section{需求分析}
   本程序可以完成两个矩阵相乘或相减的计算任务。
   输入一个矩阵的格式为
   
   
   $m$ $n$ $p$


   $i_1$ $j_1$ $val_1$


   $i_2$ $j_2$ $val_2$


   $i_3$ $j_3$ $val_3$


   ........


   $i_p$ $j_p$ $val_p$


   $m$表示矩阵的行数,$n$表示矩阵的列数,$p$表示稀疏矩阵中结点的个数。
   

   本程序要求用户输入两个矩阵A、B,然后再选择计算方式。然后程序会在后端进行计算,
   将计算结果保存在矩阵C中。
   最后程序将C输出到终端,输出格式同样使用三元组进行输出,顺序为先列后行。


   同时,我们对于用户的输入作出如下限制:
   \begin{enumerate}
      \item $n > 0$ 且 $m > 0$ 且 $p >= 0$
      \item 对于一个矩阵,数值的下标应该保持在矩阵范围内(从1开始),不得超出矩阵。
      \item 必须要输入两个矩阵后再进行运算。
   \end{enumerate}
   对于不符合输入限制要求的操作,程序将在终端上输出(Input Error)


   同时,用户应该对输入的矩阵是否能进行运算进行检查。
   如果输入的矩阵不能进行运算,程序将返回(Math Error).
   可能的原因分别有:
   \begin{enumerate}
      \item Error Code 0: 相加的两个矩阵的行数与列数不相等。
      \item Error Code 1: 两个矩阵不能相乘
   \end{enumerate}
\section{概要设计}
   程序由三个模块组成,分别为$Matrix$,$Solution$和$main$

   
   $Matrix$ 是一种数据类,定义了稀疏矩阵类以及它的运算方法,包括在矩阵中插入元素。


   $Solution$ 是可以与用户交互的类,定义了如何输入和输出,以及调用$Matrix$中的方法完成计算。


   $main$ 为程序的主函数,主要用于初始化$Solution$和调用$Solution$中与用户交互的方法。
   主程序的流程如下:
   \begin{enumerate}
      \item 初始化$Solution$
      \item 调用输入方法
      \item 调用计算方法
      \item 调用输出方法
      \item 释放内存
   \end{enumerate}

\newpage

\section{详细设计}
\begin{algorithm}[htb] 
   \caption{ Solution结构定义 } 
   \label{alg:Framwork} 
   \begin{algorithmic}[1]
      \State 读入需要被操作的矩阵
      \Function {readMatrix}{void}
         \State 初始化矩阵A
         \State 读入m,n,size
         \For {$1 \to size$}
            \State 读入三元组,并插入到矩阵A中
         \EndFor

         \State 初始化矩阵B
         \State 读入$m$,$n$,$size$
         \For {$1 \to size$}
            \State 读入三元组,并插入到矩阵B中
         \EndFor
      \EndFunction
      \State 读入运算类型并得出运算结果
      \Function{operate}{void}
         \State 从用户处读入运算类型,保存在op中
         \If {$op = 0$}
            \State   result = X + Y
         \Else
            \State   result = X * Y
         \EndIf
      \EndFunction
      \State 打印运算结果
      \Function{printMatrix}{void}
         \State 输出Result矩阵
      \EndFunction

   \end{algorithmic} 
\end{algorithm}

\newpage

\begin{algorithm}[htb] 
   \caption{ Matrix结构定义 } 
   \label{alg:Framwork} 
   \begin{algorithmic}[1]
      \State 将一个三元组插入到矩阵中
      \Function {insert}{Tuple node}
         \State 检查Matrix是否超限
         \For{Matrix所有的三元组}
            \If{第i个三元组与node在同一个位置}
               \State 将node的数值加到当前三元组数值上
            \EndIf
         \EndFor
         \If{Matrix中没有该三元组}
            \State 在三元组尾部插入node
         \EndIf
      \EndFunction
      \Function{printMatrix}{Matrix X}
         \State 输出Result矩阵
      \EndFunction

   \end{algorithmic} 
\end{algorithm}

\newpage

\begin{algorithm}[htb] 
   \caption{ Matrix结构定义 } 
   \label{alg:Framwork} 
   \begin{algorithmic}[1]
      \State 将当前矩阵与矩阵X相加,返回结果
      \Function{operator +}{Matrix X}
         \State 检查是否符合相加的条件
         \State 初始化result矩阵,保存计算结果
         \State 将log1指向左矩阵的头,log2指向右矩阵的头
         \While{log1没有达到尾部||log2没有达到尾部}
            \If{log1达到了尾部}
               \State 将log2指向的元素加入result并递增log2
            \EndIf
            \If{log2达到了尾部}
               \State 将log1指向的元素加入result并递增log1
            \EndIf
            \If{log1指向的三元组与log2指向的三元组位置相同}
               \State 将两个三元组相加后加入到result,同时递增两个指针
            \EndIf
            \If{log1指向的三元组在log2前面}
               \State 将log1指向的元素加入result并递增log1
            \Else
               \State 将log2指向的元素加入result并递增log2
            \EndIf
         \EndWhile
      \EndFunction

   \end{algorithmic} 
\end{algorithm}

\newpage

\begin{algorithm}[htb] 
   \caption{ Matrix结构定义 } 
   \label{alg:Framwork} 
   \begin{algorithmic}[1]
      \State 将当前矩阵右乘矩阵X,返回结果
      \Function{operator *}{Matrix X}
         \State 判断两矩阵是否可以相乘
         \State 将cur指向左矩阵的开头
         \State 初始化矩阵result保存结果
         \For{$i = 1 \to n$}
            \State 递增$cur$直到$cur \leq i$
            \For{$j = 1 \to m$}
               \State 初始化temp三元组,保存乘法的结果
               \For{$index = cur$其中index小于边界且index处三元组的行等于i}
                  \For{$index\_x = 1 to size$}
                     \If{三元组index\_x可以与三元组index相乘}
                        \State 将三元组index\_x与三元组index相乘,结果加入到temp中
                     \EndIf
                  \EndFor
               \EndFor
               \If{判断temp结果非0}
                  \State 将temp插入到result中
               \EndIf
            \EndFor
         \EndFor
               
      \EndFunction

   \end{algorithmic} 
\end{algorithm}

\newpage

\section{调试分析报告}

   在本次实验中我们继续沿用之前设计的Vector类。
   因为我们对矩阵相乘后的结果矩阵的大小未知,同时为了节省内存,避免开出 n*m大小的矩阵。
   如果这样我们使用稀疏矩阵将会毫无意义。


   同时我们对计算过程做了优化,在插入AB矩阵的所有数据后,我们会对AB矩阵中的三元组进行排序。
   排序顺序表现为矩阵中个个元素的位置从前往后,即先比较行后比较列。


   在经过这些优化后,我们得到的加法算法的复杂度为$O(size_A + size_B)$。


   乘法则较为复杂,优化后,$cur$指针每次递增次数的期望为$\sqrt{size_A}$,$N$与$M$表示结果矩阵的列数与行数乘法复杂度为$O(NM*\sqrt{size_A}*size_B)$

\section{用户使用说明}
输入说明:

用户先根据提示输入m,n,size,分别代表A矩阵的行数,列数与非空元素个数;

再输入size行,每行输入非空元素的坐标i,j与值num。

根据提示重复上述操作,输入B稀疏矩阵。(每行的数字用空格分隔)

再输入操作码(0/1),0代表A、B矩阵相加,1代表A、B矩阵相乘。

输出说明:

若不能进行运算,则输出"B Matrix can't add to A."或"B Matrix can't multiply to A."

若能进运算,则输出的第一行为运算后矩阵的行m,列n,与非空元素个数size。

后面输出size行代表每个非空元素的坐标i,j与值num。

用户可以使用IDE或者手动编译源代码main.cpp,获得可执行文件。

另外,笔者使用的gcc版本为8.1.0。   

运行可执行文件后,用户可以选择文件输入或者交互式输入(1/0)。

在选择文件输入模式下,输入将重定向到in.txt,输出将重定向到output.txt文件。

结束程序可以输入EOF

\section{测试结果}
用于测试的样例如下:

Sample1和2对应矩阵乘法加法失效的情况

Sample3和4对应矩阵乘法加法正常运算的结果

Sample 1:

   Input:

   3 4 2\\
   2 1 21\\
   3 4 34\\
   3 4 3\\
   1 2 12\\
   2 1 21\\
   3 2 32\\
   1\\


   Output:\\
   B Matrix can't multiply to A.


Sample 2:

   Input:\\
   2 3 3\\
   1 1 11\\
   1 3 13\\
   2 2 22\\
   3 2 3\\
   1 1 66\\
   2 3 88\\
   3 3 33\\
   0


   Output:\\
   B Matrix can't add to A.


Sample 3:

   Input:\\
   2 2 2\\
   1 1 11\\
   2 2 22\\
   2 2 2\\
   1 2 12\\
   2 1 21\\
   1


   Output:

   2 2 4
   1 1 11\\
   1 2 12\\
   2 1 21\\
   2 2 22\\

Sample 4:

   Input:
   3 3 5\\
   1 1 11\\
   1 3 13\\
   2 2 22\\
   3 2 32\\
   3 3 33\\
   3 3 3\\
   1 1 66\\
   2 3 88\\
   3 3 33\\
   0


   Output:\\
   1 1 77\\
   1 3 13\\
   2 2 22\\
   2 3 88\\
   3 2 32\\
   3 3 66